\documentclass{beamer}

\usepackage[utf8]{inputenc}
\usepackage[T1]{fontenc}
\usepackage[polish]{babel}
\usepackage[colorinlistoftodos]{todonotes}
\usepackage{hyperref}
\usepackage{subfig}

\newtheorem{lemat}{Lemat}
\newtheorem{tw}{Twierdzenie}
\newtheorem{przyklad}{Przykład}
\newtheorem{zad}{Zadanie}
\theoremstyle{definition}
\newtheorem{df}{Definicja}
\newtheorem{alg}{Algorytm}
\theoremstyle{remark}
\newtheorem{uwaga}{}

\title{Maszyna Turinga}
\author{Tomasz Janiszewski}
\date{}
\institute{MiNI PW}
\usetheme{Warsaw}


\begin{document}
\begin{frame}
\titlepage
\end{frame}
\section{Maszyna Turinga}

\begin{frame}
	\begin{df}[Maszyna Turinga - model podstawowy]~\\
	Maszyną Turinga w modelu podstawowym nazywamy system
	\begin{equation}
	M = (Q, \Sigma, \Gamma, \delta, q_0, B, F)
	\end{equation}
	\begin{description}
	\item[$Q$] -- skończony zbiór stanów
	\item[$\Sigma$] -- zbiór symboli wejściowych
	\item[$\Gamma$] -- skończony zbiór wszystkich symboli będący alfabetem taśmy
	\item[$\delta$] -- funkcja przejścia $\delta = Q\times\Gamma \rightarrow Q\times\Gamma\times\{L,P\}$
	\item[$q_0$] -- stan początkowy $q_0 \in Q$
	\item[$B$] -- zdefiniowany symbol pusty $B\in\Gamma$
	\item[$F$] -- zbiór stanów akceptujących $F\subset Q$
	\end{description}
	\end{df}	
\end{frame}

\begin{frame}
	\begin{uwaga}
		Obliczenie Maszyny Turinga polega na wykonaniu kolejnych ruchów określonych funkcją przejścia
	\end{uwaga}
\end{frame}

\begin{frame}
	\begin{uwaga}
		Obliczenie może się zakończyć ale może też być kontynuowane w nieskończoność
	\end{uwaga}
\end{frame}

\begin{frame}
	\begin{df}[Relacja ruchu]
		Obliczeniem Maszyny Turinga nazywamy ciąg opisów chwilowych pozostających w relacji okerślonej funkcją przejścia
	\end{df}
\end{frame}

\begin{frame}
\begin{df}[Maszyna Turinga z wartownikiem]~\\
Maszyną Turinga z wartownikiem nazywamy system $M = (Q, \Sigma, \Gamma, \delta, q_0, B, F)$
z tym że alfabet taśmy posiada specjalny symbol $\#$ nazywany wartownikiem, zapisany w pierwszej
komórce taśmy.
\end{df}
\begin{uwaga}
Ta maszyna jest równoważna Maszynie Turinga w modelu podstawowym
\end{uwaga}
\end{frame}

\begin{frame}
	\begin{df}[Maszyna Turinga ze stanem akceptującym]~\\
	Maszyną Turinga z własnością stopu ze stanem akceptującym nazywamy system
		\begin{equation}
			M = (Q, \Sigma, \Gamma, \delta, q_0, B, F = \{ACC\})
		\end{equation}
	Przy czym istnieje założenie, że przejście do stanu akceptującego jest zakończeniem obliczeń
	\end{df}
\end{frame}

\begin{frame}
\begin{df}[Maszyna Turinga z własnością stopu]~\\
Maszyną Turinga z własnością stopu nazywamy system
\begin{equation}
M = (Q, \Sigma, \Gamma, \delta, q_0, B, F = \{ACC\}, N = \{REJ\})
\end{equation}
gdzie $N$ to zbiór stanów odrzucających.
\begin{itemize}
\item Zakończenie obliczeń jest równoważne przejściu do stanu akceptującego lub odrzucającego
\item Maszyna kończy każde swoje obliczenie
\end{itemize}
\end{df}
\end{frame}

\begin{frame}
	\begin{uwaga}
	Maszyna Turinga z własnością stopu nie akceptuje wszystkich języków akceptowanych przez 
	Maszynę Turinga w modelu podstawowym, jest jej szczególnym przypadkiem
	\end{uwaga}
\end{frame}

\begin{frame}
\begin{df}[Maszyna Turinga z taśmą wielościeżkową]~\\
Maszyną Turinga z taśmą wielościeżkową nazywamy system $M = (Q, \Sigma, \Gamma, \delta, q_0, B, F)$,
gdzie funkcja przejścia zdefiniowana jest następująco
\begin{equation}
\delta : Q\times\Gamma^k\rightarrow G\times\Gamma\times\{L,P\}
\end{equation}
\end{df}
\end{frame}

\begin{frame}
\begin{tw}
Maszyna Turinga z taśmą wielościeżkową jest równoważna Maszynie Turinga w modelu podstawowym (jednościeżkowej)
\end{tw}
\begin{uwaga}
Można ją zastąpić maszyną jednościeżkową kosztem zwiększenia alfabetu taśmy, który będzie
iloczynem kartezjańskim alfabetów ścieżek.
\end{uwaga}
\end{frame}

\begin{frame}
\begin{tw}
Maszyny Turinga z taśmą obustronnie nieograniczoną są równoważne Maszyną Turinga w modelu podstawowym
\end{tw}
\end{frame}

\begin{frame}
\begin{df}[Wielotaśmowa Maszyna Turinga]~\\
k-taśmową Maszyną Turinga nazywamy system
\begin{equation}
M = (Q, \Sigma, \Gamma_1\times\dots\times\Gamma_k, \delta, q_0, B, F)
\end{equation}
gdzie funkcja przejścia zdefiniowana jest następująco
\begin{equation}
\delta: Q\times(\Gamma_1\times\dots\times\Gamma_k) \rightarrow Q\times(\Gamma_1\times\dots\Gamma_k)\times\{L,R,S\}^k
\end{equation}
\end{df}
\end{frame}

\begin{frame}
\begin{tw}
Klasa Maszyn Turinga wielotaśmowych jest równoważna klasie Maszyn Turinga w modelu podstawowym
\end{tw}
\begin{uwaga}
Symulacja maszyny wielotaśmowej jednotaśmową skutkuje wzrostem złożoności obliczeniowej proporcjonalną do kwadratu liczby ruchów.
\end{uwaga}
\end{frame}

\begin{frame}
\begin{df}[Niedeterministyczna Maszyna Turinga]~\\
Definicja jest analogiczna do Maszyny Turinga w modelu podstawowym z tym że funkcja przejścia
zdefiniowana jest następująco
\begin{equation}
\delta: Q\times\Gamma \rightarrow \bigcup^\infty_{k=0} (Q\times\Gamma\times\{L,R\})^k\quad
\footnote{Dla $k=0$ wartość jest nieokreślona}
\end{equation}
Zatem ruch tej maszyny polega na niedeterministycznym wyborze jednej z wartości funkcji przejścia,
a następnie wykonaniu ruchu takiego jaki wykonałaby deterministyczna maszyna.
\end{df}
\end{frame}

\begin{frame}
\begin{uwaga}
Obliczenie niedeterministycznej Maszyny Turinga jest drzewem w którym
\begin{itemize}
\item wierzchołki etykietowane są opisem chwilowym maszyny
\item potomkami dowolnego wierzchołka są wierzchołki pozostające z nim w relacji ruchu
\end{itemize}
\end{uwaga}
\end{frame}

\end{document}